\section{Conclusiones y recomendaciones}

\subsection{Conclusiones}
\begin{itemize}
    \item Se logró una comprensión sólida de los microcontroladores, destacando el PIC12F675, incluyendo su arquitectura, periféricos y capacidades de programación.
    \item Se demostró la habilidad para diseñar circuitos eficientes utilizando GPIOs, LEDs y un pulsador, con un enfoque en la precisión de la lectura analógica y el manejo de rebotes.
    \item     El desarrollo del simulador de dado no solo consolidó los conceptos teóricos, sino que también permitió aplicarlos en un proyecto práctico, demostrando la utilidad y versatilidad de los microcontroladores en situaciones del mundo real.
\end{itemize}

\subsection{Recomendaciones}
\begin{itemize}
    \item Verificar los valores de los bits de inicialización de pines al igual que los bits CONFIG en el POR, ya que esta fue la causa de la mayoría de los errores que se presentaron en este laboratorio.

\end{itemize}





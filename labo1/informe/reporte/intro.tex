\section{Introducción}

\hspace{1.26cm} Este informe documenta el diseño y desarrollo de un simulador de dado utilizando el microcontrolador PIC12F675. El objetivo del laboratorio fue introducir conceptos fundamentales sobre microcontroladores y prácticas de manejo de GPIOs.El PIC12F675 es un microcontrolador de 8 bits tipo CMOS con características versátiles, incluyendo una célula de memoria FLASH de 1024 palabras, convertidor analógico a digital, comparador analógico y capacidad de programación In-Circuit™. Además, cuenta con periféricos como temporizadores y múltiples pines I/O para interactuar con el entorno externo.

\hspace{1.26cm} El diseño del circuito implicó el uso de LEDs para representar los valores del dado y un pulsador para generar la aleatoriedad en la selección de los números. Se implementó un manejo de rebotes en el pulsador para asegurar una lectura precisa. El análisis del programa reveló un diseño modular y eficiente, con subrutinas bien definidas para cada tarea específica. Se realizó una verificación electrónica del circuito para confirmar su funcionamiento correcto, evidenciando lecturas esperadas de tensión y corriente.

\textbf{Palabras clave:} microcontrolador, rebotes, dado, LED, pseudo-aleatorio, convertidor ADC
